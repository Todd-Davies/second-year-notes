\documentclass[frontgrid]{flacards}
\usepackage{color}
\usepackage{tabularx}
\definecolor{light-gray}{gray}{0.75}

\newcommand{\frontcard}[1]{\textcolor{light-gray}{\colorbox{light-gray}{$#1$}}}
\newcommand{\backcard}[1]{#1} 

\newcommand{\flashcard}[1]{% create new command for cards with blanks
    \card{% call the original \card command with twice the same argument (#1)
        \let\blank\frontcard% but let \blank behave like \frontcard the first time
        #1
    }{%
        \let\blank\backcard% and like \backcard the second time
        #1
    }%
}

\begin{document}

\pagesetup{2}{4} 

\card{
	What happens when `LDA s' is run?
}{
	ACC = [s]
}

\card{
	What happens when `STA s' is run?
}{
	[s] = ACC
}

\card{
	What happens when `ADD s' is run?
}{
	ACC += [s]
}

\card{
	What happens when `SUB s' is run?
}{
	ACC -= [s]
}

\card{
	What happens when `JMP s' is run?
}{
	PC = s
}

\card{
	What happens when `JGE s' is run?
}{
	if ACC >= 0 then PC = s
}

\card{
	What happens when `JNE s' is run?
}{
	if ACC != 0 then PC = s
}

\card{
	What three steps occur during the fetch phase?
}{
	\begin{tabularx}{0.32\textwidth}{l X}
		1. & Use PC as address to read memory\\
		2. & Save result of read in CPU\\
		3. & Increment PC
	\end{tabularx}
}

\card{
	What control signals do all registers need?
}{
	An enable signal
}

\card{
	What control signal does a multiplexer need?
}{
	A signal to select an input
}

\card{
	What control signals does the memory need?
}{
	Ren (read enable) and Wen (write enable)
}

\card{
	Which 3 signals control the ALU?
}{
	add, sub \& byp
}

\card{
	What is a process?
}{
	A program in execution, the thread + address space.
}

\card{
	What is the address space?
}{
	All memory locations the process can use.
}

\card{
	What is a thread?
}{
	A sequence of instructions that are obeyed.
}

\card{
	What is multi-threading?
}{
	This is where we have multiple threads within the same process
}

\card{
	How do we make programs think they have sole use of memory?
}{
	Use \textbf{relocation}, where we swap a program out of memory and later swap it back in.
}

\card{
	What are the three different approaches to engineering an OS?
}{
	Monolithic, layered and micro-kernels.
}

\card{
	What are the three process states?
}{
	Running, ready, blocked
}

\card{
	In the diagram, what is happening at each stage?
}{
	\begin{tabularx}{0.32\textwidth}{l X}
		1. & Process need to wait for I/O or event.\\
		2. & Process forcibly preempted - \textbf{interrupt / relinquish CPU / time-slice expired}.\\
		3. & Scheduler selects process to run.\\
		4. & I/O or event occurs.\\
	\end{tabularx}
}

\card{
	What is a PCB table?
}{
	Process control block, it contains all of the information needed about processes.
}

\card{
	In scheduling, what do the following mean?
	\begin{tabularx}{0.32\textwidth}{l X}
		1. & CPU burst\\
		2. & I/O burst\\
		3. & CPU bound\\
		4. & I/O bound\\
	\end{tabularx}
}{
	\begin{tabularx}{0.32\textwidth}{l x}
		1. & Process executing on CPU\\
		2. & Process blocked, waiting for I/O\\
		3. & Long CPU bursts\\
		4. & Short CPU bursts\\
	\end{tabularx}
}

\card{
	What is a processes turnaround time?
}{
	The time from a process being submitted to it getting completed.
}

\card{
	What is a processes waiting time?
}{
	The time that the process waits to run.
}

\end{document} 
