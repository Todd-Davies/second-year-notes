% Set the author and title of the compiled pdf
\hypersetup{
  pdftitle = {\Title},
  pdfauthor = {\Author}
}


\section{Recap of COMP12111 \& COMP15111}

There material in both the {\it Fundamentals of Computer Architecture} and the
{\it Fundamentals of Computer Engineering} courses in the first year provides a
good base for the course this year. The following re-visits that material and
builds upon it.

\subsection{Datapath and Control}

From the point of view of the CPU all data is of a fixed size, the length of one
word. Each word is usually moved around the architecture of the computer in a
bit-parallel manner, that is to say that there are at least the same number of
wires in a bus between any two components in the system as there are bits in the
word.

The individual operations on each bit inside a word when the CPU performs an
operation on the whole word are usually identical. This results in a very
regular datapath with lots of duplicated (usually by as many times as the word
length) hardware logic.

Control logic is derived after the datapath has been conceived. It governs which
operation is performed at what time, and is different for each instruction in
the instruction set.

A typical example of control logic might be to control the enable pin on a
binary adder. The datapath will direct bits to the adder all the time, but the
control logic will determine if the result is sent forward.

\subsection{The MU0 Instruction Set Architecture}

The MU0 is a very simple 16 bit word architecture, and as a result, the
instruction set is also very simple. Each instruction can address one memory
location, and consists of four bits for the instruction (allowing sixteen
instructions to be coded) and twelve bits for the memory address as illustrated
in Figure~\ref{instruction}. Since we can only store twelve bits of memory
address in the instruction, and the architecture is very simple, the system has
$2^{12}$ words of memory, which is equivalent to \SI{8}{\kilo\byte}.

\begin{figure}[ht!]
  \centering
  \includegraphics[width=90mm]{diagrams/instruction.pdf}
  \caption{A generic MU0 instruction}
  \label{instruction}
\end{figure}

The MU0 has two programmer visible registers, the Program Counter and the
Accumulator. The Program Counter stores the address in memory of the next
instruction to be executed, thus being twelve bits long. The Accumulator is
sixteen bits long, and stores the result of the last arithmetic operation.

The instructions that the MU0 understands are listed in
Table~\ref{instruction_set}.

\begin{table}[ht!]
  \centering
  \begin{tabular}{|c|c|c|}
    \hline
    {\bf Op Code} & {\bf Mnemonic} & {\bf Description}\\ \hline
    0 & LDA $[op]$ & $[op] \rightarrow Acc$\\ \hline
    1 & STO $[op]$ & $Acc \rightarrow [op]$\\ \hline
    2 & ADD $[op]$ & $Acc = Acc + [op]$\\ \hline
    3 & SUB $[op]$ & $Acc = Acc - [op]$\\ \hline
    4 & JMP $[op]$ & $PC = S$\\ \hline
    5 & JGE $[op]$ & If $Acc >= 0$ then $PC = S$\\ \hline
    6 & JNE $[op]$ & If $Acc \not= 0$ then $PC = S$\\ \hline
    7 & STP & Stop\\ \hline
  \end{tabular}
  \caption{The MU0 instruction set}
  \label{instruction_set}
\end{table}

\subsection{Maintaining Processor State}

If the execution cycle of the MU0 was somehow disrupted, say because of an
interrupt call, it would be handy to save the state of the processor before
switching to a different task (e.g. running the interrupt handler).

The way to do this is to save the registers in memory, doing the other task, and
then reloading them when it's time to resume execution of the program.

\subsection{The Fetch Execute Cycle}

The fetch-execute cycle describes how a CPU executes instructions. First, the
next instruction is fetched from memory (at the address pointed to by the PC),
then the instruction is executed. Since some instructions access memory (such as
load and store), and we can only do one memory access per clock cycle, one
fetch-execute cycle takes two clock cycles, one for fetching, and one for
execution.

\subsubsection{Fetching instructions}

Fetching is an operation that is the same for all instructions. First memory
addressed by the PC is read and stored into the Instruction Register (IR). This
is a 16 bit internal register that isn't visible to programmers. Once this has
occurred, the PC is incremented. This means that the RAM must be able to send a
word directly to the instruction register, so a datapath must be in place to
allow this.

\subsubsection{Executing instructions}

It is obvious that different instructions will have different paths of execution
within the processor, and will have different effects on components within the
system.

\paragraph{{\tt JMP}} In order to execute the {\tt JMP} instruction, the last
twelve bits are read from the instruction register and transferred over to the
PC. This means that there must be a datapath from the bottom twelve bits of the
IR to the PC.

\paragraph{{\tt STA}} When {\tt STA} is executed, the bottom twelve bits in the IR
are used direct the contents of the accumulator to a location in memory. To do
this, we need a datapath from the bottom twelve bits of the IR to the part of
the RAM that takes addresses, and from the PC to the part of the RAM that takes
data.

\paragraph{{\tt ADD}} To perform the {\tt ADD} instruction, we need to fetch the
bottom twelve bits of the IR and send it to the RAM. The result should be fed
into the adder along with the contents of the accumulator. The result of the
calculation should be sent to the accumulator. To do this, we need datapaths
from the accumulator to the ALU, the RAM to the ALU and finally from the ALU to
the accumulator.

\paragraph{Control Signals} whenever two separate components within the system
interact. For example, every time the CPU loads a word from the RAM, a control
signal must be sent to say `load', and every time the {\tt ADD} command it
executed, the ALU must be sent a control signal to say `add' as opposed to
subtract or shift.

\paragraph{Timing} Timing is very important when executing the instructions. If
the result of a load from RAM hasn't yet returned, but the control signal to the
ALU to add is sent, then the wrong result will almost certainly occur! In order
for everything to run smoothly, the critical path for each operation must be
worked out, and time allowed for signals to propagate through even the longest
critical path.

\subsubsection{Deriving the datapaths from the operation of instruction}

In order to produce a working processor, we need to look at all the instructions
that can be executed by the processor, and examine what datapaths and control
signals they require to work. Only when we have this information can we begin to
actually design the hardware on the CPU.

\marginpar{Note that data going to one destination can only go to one source, so
if you want multiple components to be able to send data to one other component,
then you must use a multiplexer with control signals in order to achieve this.}

\subsection{Control Signals}

The purpose of control signals is to make each component within the CPU function
as intended for each specific instruction. Control signals include:

\begin{itemize}
  \item Enable write for registers
  \item Enable write for memory
  \item Enable read for memory
  \item Multiplexer input select
  \item ALU actions (add, subtract, bypass)
\end{itemize}

Sometimes, one component (such as the ALU) may have control inputs that can be
represented by more than two states (add, bypass, subtract). If this is the
case, then multiple wires (a bus) is used to specify its action.

The first lab in the course shows the control signals sent for each instruction,
the solution for which is shown in Tables~\ref{lab:1:fetch} and
\ref{lab:1:execute}.

\begin{table}[!ht]
  \centering
    \begin{tabular}{|l|l|l|}\hline
    En\_IR  & 1 &(enable write to IR)\\ \hline
    En\_PC  & 0 &(enable write to PC)\\ \hline
    En\_ACC & X &(enable write to ACC)\\ \hline
    \hline
    byp & X &(ALU action: bypass)\\ \hline
    add & X &(ALU action: add)\\ \hline
    sub & X &(ALU action: subtract)\\ \hline
    \hline
    Ren & 1 &(RAM action: Read)\\ \hline
    Wen & 0 &(RAM action: Write)\\ \hline
    addr\_Mux& 1  &(RAM address = PC, otherwise IR.S)\\ \hline
  \end{tabular}
  \caption{The control signals in the MU0 fetch phase}
  \label{lab:1:fetch}
\end{table}

\begin{table}[!ht]
  \centering
  \begin{tabular}{|l|l|l|l|l|l|l|l|}\hline
      & lda & sta & add & sub & stp & jump  & no jump\\ \hline
    \hline
    En\_IR   & 0  & 0 & 0 & 0 & 0 & 0 & 0 \\ \hline
    En\_PC   & 0  & 0 & 0 & 0 & 0 & 1 & 0 \\ \hline
    En\_ACC  & 1  & 0 & 1 & 1 & 0 & 0 & 0 \\ \hline
    \hline
    byp      & 1  & X & 0 & 0 & X & X & X \\ \hline
    add      & 0  & X & 1 & 0 & X & X & X \\ \hline
    sub      & 0  & X & 0 & 1 & X & X & X \\ \hline
    \hline
    Ren      & 1  & 0 & 1 & 1 & 0 & 1 & 0 \\ \hline
    Wen      & 0  & 1 & 0 & 0 & 0 & 0 & 0 \\ \hline
    addr\_Mux& 0  & 0 & 0 & 0 & X & 0 & 1 \\ \hline
  \end{tabular}
  \caption{The control signals in the MU0 execute phase}
  \label{lab:1:execute}
\end{table}

\section{What does an Operating System do?}

The job of an operating system varies from system to system but on general, it
is responsible for managing the resources of the system (including dealing with
concurrency, security etc) and abstracting the implementation of the system from
the running programs (such as what exact components are being utilised).

\subsection{Processes}

A process is a program that is currently running on the system. It consists of a
Thread (a set of instructions to be executed) and address space (a set of memory
locations that can be accessed by the thread). In most systems, multi-threading
is used to allow each process own multiple threads, and therefore execute in
parallel.

\marginpar{Nearly all systems have many processes running at any one time, on a
Linux system, use {\tt htop} or {\tt ps aux} to see what processes are running.}

\subsection{Address Space}

\textit{Address space} (aka memory space) is a term used to speak about a
section of memory. This could be the whole memory available to the system, the
memory that a specific program has access to etc.

When a program starts, it assumes that it does so from memory address {\tt 0}.
On a single process system this is okay, however this presents a problem on
systems where multiple processes run concurrently, since no two processes can
share the same memory space.

Sometimes, operating systems may even running pause programs, move them out of
memory (onto secondary memory such as hard drive) and later on swap it back in
at a different place in memory.

In both cases, a technique called {\it Relocation} is used to make every running
program able to safely assume that it has sole use of memory.

In order to facilitate relocation, operating systems abstract away the
implementation of the hardware, and instead provide a virtual machine for each
program. This enables programs to behave as though they have the whole system
to themselves, and it also lets the operating system easily stop programs
interfering with each other (such as providing disjoint memory spaces for each
program).

\subsection{Modes of operation}

It is often necessary to prevent some programs from executing some operations,
such as manipulating memory, or allocating CPU time. In order to achieve this,
operating systems nearly always implement different `modes' of operation that
processes can run under. The two most common modes are \textit{user} and
\textit{system}.

All the processes owned by the operating system will run under system mode,
which is very permissive and lets programs perform operations with the
potential for misuse. Programs that the user might run are usually executing
under user mode. User mode is less permissive, and restricts certain
operations, yet the restricted operations aren't usually required for normal
programs.

\subsubsection{System calls}

If there was an eventuality where a program needed to perform privileged
operation that wasn't permitted under it's current mode, then it can use a
system call to achieve the same result. The premise is that the operating
system will provide a `gatekeeper' function that will perform the requested
operation, but only after the parameters have undergone checks to ensure that
the application isn't behaving badly. The execution of the user program will
(of course) pause while the system call is running.

Lots of functions in languages that you already know might just be wrappers
around system calls, albeit often with slightly more functionality. The course
notes make a good example; \texttt{fread} in C uses the \texttt{UNIX} system
call \texttt{read}.

\section{Engineering an Operating System}

Like a lot of things in Computer Science, operating systems can often be
conceptualised as being built up of several layers. As you inspect further and
further into the OS, looking deeper into each layer as you go, the level of
abstraction decreases.

The outermost layer could be seen to be the UI, which is obviously a very
abstracted way of thinking about a computer. The kernel is probably the lowest
layer, since the details of how the hardware is managed is, also fairly
obviously, a very low level of abstraction.

Different operating systems can contain a different number of layers of
abstraction so that they are best suited to their purpose. Some operating
systems will contain little more than a microkernel, which will have the bare
minimum of logic required to keep a computer running.

Some components of an operating system are monolithic, most notably the Linux
kernel. A monolithic component is easier to design (especially for a kernel)
since there is less inter-process communication to worry about, and trap falls
like race conditions can be more easily avoided when everything is running on
a single thread.

\marginpar{
  \begin{sloppypar}
    See this StackOverflow answer for an explanation (and follow the
    link to the Linus V Tanenbaum showdown on the topic):
    \texttt{http://stackoverflow.com/questions}
    \texttt{/1806585/why-is-linux-called}
    \texttt{-a-monolithic-kernel}
  \end{sloppypar}
}

I feel obliged to point out that monolithic, in this instance, refers to the
fact that the program is running under \textit{one} process that is in system
mode. It doesn't mean that the code is all in one big file, or even in one big
project, since many monolithic projects employ some degree of modularity in
the development process at least.

\subsection{Managing processes}

When operating systems were first developed, they usually ran one process at a
time. This was obviously limiting for the users, since only one person could do
stuff at a time. In light of this, timesharing was developed, which is a way for
the computer to be used by multiple users at once, even here, users would all
share one CPU and one PC though.

Now, processes execute within their own virtual CPU, while the real CPU switches
back and forth giving time to each task. This is good, since the order with
which the CPU gives it's time to processes can be used to effectively make some
processes run faster than others.

We have already learnt that a process is made up of a thread and some address
space, \marginpar{(Processes also have register values and external interfaces
too, but they're not as important)} but we've not been over how a process is
created. In order to do this, an existing process must create new child process.

Processes can be killed by themselves (with a variety of different exit codes),
by other processes (using something like the \texttt{kill}) command, or in some
systems, by the parent process terminating.

The lifecycle of a process, and the states it occupies can be represented in a
state machine like diagram, as shown in Figure~\ref{fig:proc-lifecycle}.

\begin{figure}[ht!]
  \centering
  \includegraphics[width=\textwidth]{diagrams/process-lifecycle.pdf}
  \caption{The lifecycle of a process}
  \label{fig:proc-lifecycle}
\end{figure}

\subsubsection{Scheduling}

One particularly troublesome section of the lifecycle of a process is going from
the \textit{ready} state to the \textit{running} state (and vice versa). In
order to do this, a process to be resumed must be chosen from a pool of
processes waiting for CPU time, and the registers, IO operations etc must be
loaded and made ready.

In order to do this, a PCB (Process Control Block) table is maintained, which
contains all the necessary information for each process to be paused and
resumed. This includes (among other things):

\begin{itemize}
  \item Process ID
  \item Parent ID
  \item Saved registers
  \item Memory, IO management information
  \item CPU scheduling info
\end{itemize}

On some operating systems, timeslices are given to processes, where on others,
they are assigned to threads. Henceforth, processes can often be made more
efficient by using threads in order to maximise the use of their timeslice
(switching to non-blocked threads whenever there is a block).

In order to handle the complicated tasks of scheduling, operating systems have a
dedicated component as part of the process manager to do this. It's main job is
decide which process should run next on what core of the CPU, while minimising
the average wait and turnaround times for processes to execute.

Processes alternate between CPU (expensive computation) and IO (blocked on IO)
bursts. Processes that have long CPU bursts are said to be CPU bound, while
processes that use lots of IO are said to be IO bound. Processes can change
their characteristics while running (if a web browser is using lots of cached
content, it might be IO bound, but if it's rendering 4K video with JavaScript,
it's probably CPU bound).

Here are some scheduling algorithms:

\begin{description}
  \item \textbf{First Come First Serve (FCFS)}\\
    This is when processes get all of the CPU time until they are finished, or
    they block. It's easy to implement, since all you need is a queue of
    processes:

    \begin{figure}[ht!]
      \centering
      \includegraphics[width=90mm]{diagrams/fcfs.pdf}
      \caption{What an FCFS implementation might behave like}
      \label{fcfs}
    \end{figure}

    FCFS is a \textbf{non-preemptive} algorithm, since processes are allowed to
    run until they finish with the CPU and start doing IO stuff, or exit.

    \marginpar{`Time slice' is also known as `Time Quantum' and it's length is
    very important. Smaller time slices are affected greater by the time it
    takes to perform a context switch. The best solution is to try and decrease
    the cost of context switching and increase the time slice length a little
    (though not too much, otherwise you end up with psudo-FCFS)}

  \item \textbf{Round Robin}\\
    The round robin algorithm is a \textbf{preemptive} algorithm. Each process
    is given a time slice when it starts processing on the CPU, and if it hasn't
    finished by the end of it's time slice, then the CPU is given to another
    process. It's still a very simple algorithm, and it's much more efficient
    than FCFS.

    \begin{figure}[ht!]
      \centering
      \includegraphics[width=90mm]{diagrams/roundrobin.pdf}
      \caption{What a Round Robin implementation might behave like}
      \label{roundrobin}
    \end{figure}

  \item \textbf{Shortest job first}\\
    We could just run whichever process will have the shortest CPU burst first.
    The problem with this one, is it can be hard to tell how long processes will
    use the CPU for. If we can pull this off though, the average turnaround time
    and wait time are very low. This would be a \textbf{non-preemptive}
    approach, since when a process starts to run, it is left running until it
    finishes it's CPU burst.

    You can also \textbf{make SJF preemptive}. In order to do this, when a new
    process is added to the queue, if it has a lower CPU burst than the
    remaining time on the currently running process, then context switch and run
    the new one first. This is called \textbf{STRF} (Shortest Time Remaining
    First).

    One problem with SJF, is that if a process has a very long CPU burst, then
    it may be starved of CPU time by the scheduler, since other processes, with
    a smaller initial burst will be allowed in first.

  \item \textbf{Priority Queue}\\
    We could have multiple queues, each with a different priority. Processes in
    higher priority queues could have longer time slices, or could finish all
    their CPU bursts before allowing other processes to start with theirs. We
    could still end up starving low priority processes using (particularly) the
    latter method, however, if we had the option to \textit{dynamically move
    processes between queues}, then we could solve this problem too.

    \marginpar{Can you work out the actual amount of CPU/IO/idle that the
    processes have? It's in the \LaTeX~source if you want to find out!}

    \begin{figure}[ht!]
      \centering
      \includegraphics[width=90mm]{diagrams/priority.pdf}
      \caption{If process A and B were in the high priority queue, process C was
      in normal priority, while D and E were in low priority, the scheduler may
      behave like this.}
      \label{priority}
    \end{figure}

    Dynamically changing which queue processes are in sounds hard, but it's not
    too bad. Every time a process uses up all of it's timeslice, then move it
    down a queue, every time it finishes it's CPU burst before it's timeslice
    expires, move it up a queue. In this manner, IO bound applications will tend
    to execute faster, and there will be less idle time. \marginpar{This isn't
    really a good thing, since your IO programs (such as a backup utility), may
    not be the ones you want to run the fastest.}

    In earlier versions of Linux\footnote{
      If you're ahead of schedule with your revision, take a look at:
      \begin{itemize}
        \item \url{http://en.wikipedia.org/wiki/O(n)_scheduler}
        \item \url{http://en.wikipedia.org/wiki/O(1)_scheduler}
        \item \url{http://en.wikipedia.org/wiki/Completely_Fair_Scheduler}
      \end{itemize}
    }, there were three queues, but each used a
    different strategy; the highest priority queue was FIFO, the next queue used
    a round robin technique and the lowest priority queue used a method where
    the timeslice differs based on the process priority.

    The low priority queue is most interesting; each process has a quantum
    associated with it, which is reduced by one for each clock tick it has on
    the CPU. The timeslice it gets the process priority plus the remaining
    quantum divided by two. When all of the processes in the queue run out of
    quantums, then the quantum of all the processes in the queue is recomputed.

    Unfortunately, the old Linux way of scheduling didn't scale well, since it
    had an $O(n)$ runtime, so when you started up a lot of processes, the
    algorithm's performance would degrade. Now, Linux only uses algorithms with
    a $O(1)$ runtime, so that the number of processes doesn't impact on the CPU
    speed.

\end{description}

The above algorithms are targeted at desktop PC's, and `normal' operating
systems, but for applications that are a little different, such as real time
systems, then different algorithms are used (maybe one that picks the process
with a short deadline for computation, or the smallest deadline - completion
time).

\subsubsection{Context switching}

A context switch occurs whenever the CPU pauses execution for one process and
resumes it for another. It's sometimes and expensive operation, and depending on
the processor, can take anywhere from one micro-second to one nano-second.

\marginpar{It is important to remember that there are plenty of things that do
need to be included in the context switch when two threads of the same process
are swapping, including registers (including the PC), and the stack.}

It is important to realise that not all context switches are equal even on the
same machine. If the processor is switching between two threads of the same
process, it won't have to switch in most of the information about memory, since
the threads will use the same memory space. Because of this, the memory cache
might also be more efficient due to locality of reference.

