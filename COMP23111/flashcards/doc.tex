\documentclass[frontgrid]{flacards}
\usepackage{color}
% For funky database symbols
\usepackage{newlfont}

\definecolor{light-gray}{gray}{0.75}

\newcommand{\frontcard}[1]{\textcolor{light-gray}{\colorbox{light-gray}{$#1$}}}
\newcommand{\backcard}[1]{#1} 

\newcommand{\flashcard}[1]{% create new command for cards with blanks
    \card{% call the original \card command with twice the same argument (#1)
        \let\blank\frontcard% but let \blank behave like \frontcard the first time
        #1
    }{%
        \let\blank\backcard% and like \backcard the second time
        #1
    }%
}

\begin{document}

\pagesetup{2}{4} 

\card{
	What is $\sigma$?
}{
	The selection operator (selects rows).
}

\card{
	What is $\pi$?
}{
	The projection operator (selects columns).
}

\card{
	What is $\delta$?
}{
	The distinct operator (makes sure no rows are repeated).
}

\card{
	What is $\times$?
}{
	The product operator (produces all permutations of the rows of two tables).
}

\card{
	What is $\Join$?
}{
	The join operator (natural or otherwise, it joins two tables together based
	on a column).
}

\card{
	What is $\rho$?
}{
	The rename operator (renames column names).
}

\end{document} 
