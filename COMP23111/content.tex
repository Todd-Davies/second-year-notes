% Set the author and title of the compiled pdf
\hypersetup{
  pdftitle = {\Title},
  pdfauthor = {\Author}
}

\section{An introduction to Database Management Systems}

DataBase Management Systems (DBMS's) are a type of middleware that provide a
layer of abstraction for dealing with databases. It is nearly always unnecessary
to write software from scratch that interfaces with a database, since a lot of
database operations will share a significant amount of logic.

Henceforth, a lot of the functionality required of applications that make use of
a database is placed into a DBMS, which application developers can make use and
save time. The DBMS acts as a service, that is well implemented and is able to
enforce good practices and advanced techniques such as concurrency, sharding,
recovery management and transactions.

Some advantages of using a DBMS include:

\begin{itemize}
  \item It decouples data inside a database from the application using it.
        Either can be re-written at any time so long as they still provide/use
        the same interface.
  \item Since the data is decoupled from the application, using a DBMS (in
        theory) lowers the development cost of the application.
  \item Most DBMS are scalable, concurrent, fault tolerant, authorisation
        control (often role based for organisations).
\end{itemize}

Even though the DBMS aims to provide a layer of abstraction for a user
application, there are several layers of abstraction within the DBMS itself.
These are:

\begin{tabularx}{\textwidth}{>{\bfseries}l X}
  Physical & Deals with the file(s) that is written to the
             storage medium that will hold the database. Needs to know about
             file formats, indexing, compression, etc.\\
  Logical  & Mainly concerned with mapping the raw data into database
             `concepts' such as tables, views etc. It is here that the formal
             specification of the database is defined, commonly used models
             include \textit{relational, XML based and document based}\\
  View     & Ensures that only authorised people can view the data.\\
\end{tabularx}

If the database is using a relational format, then it will be defined by a
schema. A schema dictates how the database is formatted; what tables there are,
and what datatypes their columns take. An instance of a database is the content
(data) inside of the database at a particular point in time. There is a certain
isomorphism between relational databases and imperative programming languages; a
schema would be akin to the declaration of variables (i.e. their names and
types), while the instance would be their values at a particular point in the
program's execution.

Irrespective of what logical model a database uses, most DBMS use between one
and three languages to interface with a user/application. These are:

\begin{itemize}
  \item Data Definition Language - used for specifying schema.
  \item Data Manipulation Language - used for mutating the data in the database.
  \item Data Query Language - used to access data in the database.
\end{itemize}

Often DBMS languages will be both a DML and a DQL, and sometimes a DDL too! One
such example is SQL, does all of the above!

\section{Relational Algebra and SQL}

Relational algebra is designed for modelling data stored in relational databases
(i.e. tables) and defining queries on it. It can perform unary operations (such
as growing, shrinking and selecting from tables), or binary operations (union,
intersection, difference, product, join).

\subsection{Selection $\sigma$}

The $\sigma$ operator can select rows that meet a certain criteria from the
table.

\begin{center}
  \begin{tabular}{lll}
    \multicolumn{3}{c}{\textbf{Alcohol-Selection}}\\
    {Type} & {Strength} & {Colour}\\ \hline
    Wine          & 11                & Red\\
    Beer          & 4.2               & Yellow\\
    Wine          & 12.8              & White\\
    Port          & 18                & Carmine\\
    Ale           & 11                & Red\\
  \end{tabular}
\end{center}

If we run $\text{\it Fine-Wines}:=\sigma_{Type=Wine}(\text{\it Alcohol-
Selection})$, we'll end up with:

\begin{center}
  \begin{tabular}{lll}
    \multicolumn{3}{c}{\textbf{Fine-Wines}}\\
    {Type} & {Strength} & {Colour}\\ \hline
    Wine          & 11                & Red\\
    Wine          & 12.8              & White\\
  \end{tabular}
\end{center}

\subsection{Projection $\pi$}

The $\pi$ operator can select rows instead of columns. If we do $\text{\it
Anonymous-Drinks} := \pi_{Strength, Colour}(\text{\it Alcohol-Selection})$:

\begin{center}
  \begin{tabular}{ll}
    \multicolumn{2}{c}{\textbf{Anonymous-Drinks}}\\
    {Strength} & {Colour}\\ \hline
    11         & Red\\
    4.2        & Yellow\\
    12.8       & White\\
    18         & Carmine\\
  \end{tabular}
\end{center}

Notice that both the 11\% Ale and the 11\% Red Wine have the same strength and
colour values. Consequently, the projection operator combines those rows into
one so the same result isn't displayed twice.

Projection can also be used to do simple arithmetic, $\text{\it Test} :=
\pi_{Strength + Strength->DStrength, Colour}(\text{\it Anonymous-Drinks})$:

\begin{center}
  \begin{tabular}{ll}
    \multicolumn{2}{c}{\textbf{Test}}\\
    {DStrength} & {Colour}\\ \hline
    22         & Red\\
    8.4        & Yellow\\
    25.6       & White\\
    36         & Carmine\\
  \end{tabular}
\end{center}

\subsection{Product $\times$}

\begin{center}
  \begin{tabular}{lll}
    \multicolumn{3}{c}{\textbf{Shops}}\\
    {Name}           & {Dodginess} & {Price}\\ \hline
    Ali's            & High        & Medium \\
    Tesco            & Low         & Medium \\
    New Zeland Wines & X.High      & Low    \\
  \end{tabular}
\end{center}

We could do a cross product with the Shops and the Alcohol-Selection tables
$\text{\it Grog-Shops} := \text{\it Shops} \times \text{\it Alcohol-Selection}$:

\begin{center}
  \begin{tabular}{llllll}
    \multicolumn{6}{c}{\textbf{Grog-Shops}}\\
    {Name}           & {Dodginess} & {Price} & {Type} & {Strength} & {Colour}\\ \hline
    Ali's            & High        & Medium  & Wine          & 11                & Red\\
    Ali's            & High        & Medium  & Beer          & 4.2               & Yellow\\
    Ali's            & High        & Medium  & Wine          & 12.8              & White\\
    Ali's            & High        & Medium  & Port          & 18                & Carmine\\
    Ali's            & High        & Medium  & Ale           & 11                & Red\\
    Tesco            & Low         & Medium  & Wine          & 11                & Red\\
    Tesco            & Low         & Medium  & Beer          & 4.2               & Yellow\\
    Tesco            & Low         & Medium  & Wine          & 12.8              & White\\
    Tesco            & Low         & Medium  & Port          & 18                & Carmine\\
    Tesco            & Low         & Medium  & Ale           & 11                & Red\\ 
    New Zeland Wines & X.High      & Low     & Wine          & 11                & Red\\
    New Zeland Wines & X.High      & Low     & Beer          & 4.2               & Yellow\\
    New Zeland Wines & X.High      & Low     & Wine          & 12.8              & White\\
    New Zeland Wines & X.High      & Low     & Port          & 18                & Carmine\\
    New Zeland Wines & X.High      & Low     & Ale           & 11                & Red\\
  \end{tabular}
\end{center}

\subsection{Renaming $\rho$}

The notation for renaming columns is pretty simple; $\text{\it Drinks} :=
\rho_{Name, Strength, Hue}(\text{\it Alcohol-Selection})$

\begin{center}
  \begin{tabular}{lll}
    \multicolumn{3}{c}{\textbf{Drinks}}\\
    {Name} & {Strength} & {Hue}\\ \hline
    Wine          & 11                & Red\\
    Beer          & 4.2               & Yellow\\
    Wine          & 12.8              & White\\
    Port          & 18                & Carmine\\
    Ale           & 11                & Red\\
  \end{tabular}
\end{center}

\subsection{Join $\Join$}

If we had:

\begin{center}
  \begin{tabular}{lll}
    \multicolumn{2}{c}{\textbf{People}}\\
    {Name} & {Drinks}\\ \hline
    Alice  & Wine\\ 
    Bob    & Beer\\ 
  \end{tabular}
\end{center}

We could join it with the Grog-Shops table, using $\text{\it Fave-Shops} :=
People \Join_{People.Drinks=Grog-Shops.Type}(\text{\it Grog-Shops})$

\begin{center}
  \begin{tabular}{lllllll}
    \multicolumn{7}{c}{\textbf{Fave-Shops}}\\
    {Person.Name} & {Grog-Shops.Name}& {Dodginess} & {Price} & {Type} & {Strength} & {Colour}\\ \hline
    Alice         &  Ali's            & High        & Medium  & Wine          & 11                & Red\\
    Bob           &  Ali's            & High        & Medium  & Beer          & 4.2               & Yellow\\
    Alice         &  Ali's            & High        & Medium  & Wine          & 12.8              & White\\
    Alice         &  Tesco            & Low         & Medium  & Wine          & 11                & Red\\
    Bob           &  Tesco            & Low         & Medium  & Beer          & 4.2               & Yellow\\
    Alice         &  Tesco            & Low         & Medium  & Wine          & 12.8              & White\\
    Alice         &  New Zeland Wines & X.High      & Low     & Wine          & 11                & Red\\
    Bob           &  New Zeland Wines & X.High      & Low     & Beer          & 4.2               & Yellow\\
    Alice         &  New Zeland Wines & X.High      & Low     & Wine          & 12.8              & White\\
  \end{tabular}
\end{center}

If two tables have a column of the same name, then they can be joined naturally
without specifying which columns to join explicitly.

\subsection{Distinct $\delta$}

The $\delta$ operator will ensure that no rows are duplicated.

\subsection{Chaining operators}

Just like in normal algebra, you can chain operators:

$\delta(\pi_{Strength, Colour}(\text{\it Alcohol-Selection}))$

Gives:

\begin{center}
  \begin{tabular}{lll}
    {Strength} & {Colour}\\ \hline
    11                & Red\\
    4.2               & Yellow\\
    12.8              & White\\
    18                & Carmine\\
  \end{tabular}
\end{center}

\section{SQL Syntax}

% Yay, this'll be fun!