\documentclass[frontgrid]{flacards}
\usepackage{color}

\definecolor{light-gray}{gray}{0.75}

\newcommand{\frontcard}[1]{\textcolor{light-gray}{\colorbox{light-gray}{$#1$}}}
\newcommand{\backcard}[1]{#1} 

\newcommand{\flashcard}[1]{% create new command for cards with blanks
    \card{% call the original \card command with twice the same argument (#1)
        \let\blank\frontcard% but let \blank behave like \frontcard the first time
        #1
    }{%
        \let\blank\backcard% and like \backcard the second time
        #1
    }%
}

\begin{document}

\pagesetup{2}{4} 

\card{
  What does $O(<expr>)$ mean?
}{
  The complexity (i.e. running time/space) is bounded by the $<expr>$.
}

\card{
  What does $\Theta(<expr>)$ mean?
}{
  The complexity (i.e. space/running time) has the complexity proportional to
  $<expr>$.
}

\card{
  What does $\Omega(<expr>)$ mean?
}{
  The complexity (i.e. running time/space) is \textit{at least} by the $<expr>$.
}

\card{
 
 }

%TODO: Add ones on the math review in the complexity section of the algorithms
%      text book (~page 25)

\card{
  Summation identity...
}{
  What it means... 
}

\end{document} 
