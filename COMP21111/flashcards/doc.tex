\documentclass[frontgrid]{flacards}
\usepackage{color}

\definecolor{light-gray}{gray}{0.75}

\newcommand{\frontcard}[1]{\textcolor{light-gray}{\colorbox{light-gray}{$#1$}}}
\newcommand{\backcard}[1]{#1} 

\newcommand{\flashcard}[1]{% create new command for cards with blanks
    \card{% call the original \card command with twice the same argument (#1)
        \let\blank\frontcard% but let \blank behave like \frontcard the first time
        #1
    }{%
        \let\blank\backcard% and like \backcard the second time
        #1
    }%
}

\begin{document}

\pagesetup{2}{4} 

% Logic functions (truth tables)

\flashcard{
  The truth table for the \texttt{and} function is:
  \begin{tabular}{c|c|c}
    Input 1 & Input 2 & Input 1 \texttt{and} Input 2\\ \hline
    T & T & \blank{T}\\  \hline
    T & F & \blank{F}\\  \hline
    F & T & \blank{F}\\  \hline
    F & F & \blank{F}\\  \hline
  \end{tabular}
}

\flashcard{
  The truth table for the \texttt{or} function is:
  \begin{tabular}{c|c|c}
    Input 1 & Input 2 & Input 1 \texttt{or} Input 2\\ \hline
    T & T & \blank{T}\\  \hline
    T & F & \blank{T}\\  \hline
    F & T & \blank{T}\\  \hline
    F & F & \blank{F}\\  \hline
  \end{tabular}
}

\flashcard{
  The truth table for the \texttt{implies} function is:
  \begin{tabular}{c|c|c}
    Input 1 & Input 2 & Input 1 \texttt{implies} Input 2\\ \hline
    T & T & \blank{T}\\  \hline
    T & F & \blank{F}\\  \hline
    F & T & \blank{T}\\  \hline
    F & F & \blank{T}\\  \hline
  \end{tabular}
}

\flashcard{
  The truth table for the \texttt{bi-implication} function is:
  \begin{tabular}{c|c|c}
    Input 1 & Input 2 & Input 1 \texttt{$\iff$} Input 2\\ \hline
    T & T & \blank{T}\\  \hline
    T & F & \blank{F}\\  \hline
    F & T & \blank{F}\\  \hline
    F & F & \blank{T}\\  \hline
  \end{tabular}
}


\end{document} 
