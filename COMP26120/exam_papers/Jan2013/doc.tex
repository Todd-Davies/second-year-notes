\documentclass{article}

% Tell me about my LaTeX bad practices
\usepackage[l2tabu, orthodox]{nag}
% Make the margins wider
\usepackage[margin=1in]{geometry}
% For nice headers
\usepackage{fancyhdr}
% For line brakes in tables
\usepackage{tabularx}
% For the split environment
\usepackage{amsmath}
% For tabs in verbatim
\usepackage{moreverb}
% Means you don't have to put \\ to start a new line.
\usepackage[parfill]{parskip}
% For code listings
\usepackage{listings}
% For code listing colours
\usepackage{color}
% For images
\usepackage[pdftex]{graphicx}
% Make LaTeX pretty with better kerning etc
\usepackage{microtype}

\input{listingstyle.tex}

\pagestyle{fancyplain}

\author{Todd Davies}
\title{COMP26120 - January 2013 - Answers}
\date{\today}

\begin{document}

\rhead{COMP26120 - January 2013 - Answers}
\lhead{\today}

\maketitle

\begin{center}
	\small Please don't assume these answers are right. This is me attempting a
	past paper for revision purposes; I could have got it all wrong ;)

  I chose to answer questions 1 \& 2.
\end{center}

\section*{Question 1}

\subsection*{Part a}

\lstinputlisting[language=Java, firstline=2, lastline=12, caption={
  Iterates through the list, divides $k$ by input[i] for each iteration, and
  uses a binary search to see if the result is in the list. The binary search
  index increases by one on every iteration so that the square of items isn't
  counted.
}]{code/part1a.java}

Runs in $O(n\log n)$ (we do a binary search ($O(\log n)$) $n$ times), but
assumes that input is sorted first.

\newpage

\subsection*{Part b}

\lstinputlisting[language=Java, firstline=4, lastline=14, caption={
  Iterates through the list, keeps a count in a HashSet of how many times it's
  seen each integer. If the count equals $\frac{|input|}{2}$, then it returns
  true.
}]{code/part1b.java}

Runs in $O(n)$ time, providing the HashMap does lookups in $O(1)$ time, which
for Java ints, it should.

\newpage

\subsection*{Part c}

\section*{Question 2}

\subsection*{Part a}

\end{document}