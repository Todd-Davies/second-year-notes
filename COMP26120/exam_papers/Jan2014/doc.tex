\documentclass{article}

% Tell me about my LaTeX bad practices
\usepackage[l2tabu, orthodox]{nag}
% Make the margins wider
\usepackage[margin=1in]{geometry}
% For nice headers
\usepackage{fancyhdr}
% For line brakes in tables
\usepackage{tabularx}
% For the split environment
\usepackage{amsmath}
% For tabs in verbatim
\usepackage{moreverb}
% Means you don't have to put \\ to start a new line.
\usepackage[parfill]{parskip}
% For code listings
\usepackage{listings}
% For code listing colours
\usepackage{color}
% For images
\usepackage[pdftex]{graphicx}
% Make LaTeX pretty with better kerning etc
\usepackage{microtype}

\input{listingstyle.tex}

\pagestyle{fancyplain}

\author{Todd Davies}
\title{COMP26120 - January 2014 - Answers}
\date{\today}

\begin{document}

\rhead{COMP26120 - January 2014 - Answers}
\lhead{\today}

\maketitle

\begin{center}
	\small Please don't assume these answers are right. This is me attempting a
	past paper for revision purposes; I could have got it all wrong ;)
\end{center}

\section{Question 1}

\subsection{Part a}

\lstinputlisting[language=Java, firstline=2, lastline=12, caption={
	Add all of the pairs of integers together in the list. Do this by looping
	through the list, and looping from 0$\rightarrow$i on each
	iteration.
}]{code/part1a.java}

Since as we're iterating through the list, we are describing an arithmetic series
(we do one additions, then two, then three etc), we can use the formula
$\frac{n(1+n)}{2}$ to find how many operations we're doing. This, in the Big-Oh
notation, equates to $O(n^2)$ since constant terms are eliminated. In most
cases, the result will be found significantly faster though.

\newpage

\subsection{Part b}

\lstinputlisting[language=Java, firstline=2, lastline=11, caption={
	Add the first list to a BitSet (a very compact array of bits), then iterate
	through the second list and add items that are in the bit set to the output
	array.
}]{code/part2a.java}

The worst case runtime of this algorithm is $O(n + m)$ since it iterates through
both arrays once, and the loops aren't nested.

\newpage

\subsection{Part c}

\lstinputlisting[language=Java, firstline=5, lastline=15, caption={
	Create a new array containing the first half, and a new array containing the
	second half, then replace them insert them into the original array.
}]{code/part3a.java}

The worst case runtime of this function is $O(n)$, since it only iterates
through the array once. The space complexity is $O(2n)$, since I couldn't figure
out how to do an in-place circular shift.

\end{document}