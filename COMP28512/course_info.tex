\section*{Introduction}

Now that the mobile telephone has evolved into a powerful computer, the mobile
dimension of computing is a vital part of Computer Science. This unit will give
insights into many issues of mobile systems, including wireless communication
networks, the processing of speech, music and other real-time signals, the
control of bit-errors and maximising battery life. The techniques and software
which underlie commonplace applications of mobile computing systems, including
smart-phones, tablets, laptop computers, MP3 players and GPS satellite
navigation, will be addressed.

\section*{Aims}

Computing is becoming increasingly mobile. This unit will give insights into the
issues of mobile systems, covering mobile communications, real-time signals such
as speech, video and music, codecs, and maximising battery life.

\begin{itemize}
  \item Commonplace examples of mobile computing systems: - mobile phones; - MP3 players; - laptop computers; - PDAs; - GPS satellite navigation.
  \item Real-time signals
  \item Analogue and digital signals; - time and frequency domain representations; - sampling, aliasing, quantization; - companding; - real-time computation.
  \item Coding, decoding and compression
  \item GSM speech coding; - MP3 music, JPEG image and MPEG video coding & decoding; - error correcting codes; - communications coding schemes.
  \item Mobile communication
  \item Transmitting real-time information over wireless networks; - principles of cellular and ad-hoc networks; - Coding of multimedia signals - to increase the capacity of radio channels; - to minimise the effect of transmission errors.
  \item Maximising battery life
  \item May be addressed at many levels including: - chip design; - signal coding and processing; - medium access control; - transmit power control.
\end{itemize}