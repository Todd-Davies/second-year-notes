%xaw
%[3, 4, 1, 2, 7, 8, 5, 6]

\card{
  \textbf{Use Case: Tax system extracts tax payments from company database  which is the actor in this company system?}\\\vspace{1em}
  \begin{tabularx}{0.8\textwidth}{lX}
    1. & Company manager.\\
    2. & Employee\\
    3. & Tax System.\\
    4. & Company database.
  \end{tabularx}
}{
  \begin{tabularx}{0.8\textwidth}{lX}
     & \\
     & \\
    3. & Correct. The tax system is an external system that uses the company system.\\
     & 
  \end{tabularx}
}

%xaj
%[4, 1, 2, 3, 8, 5, 6, 7]

\card{
  \textbf{Which of the following is the odd one out in a Book ordering System?}\\\vspace{1em}
  \begin{tabularx}{0.8\textwidth}{lX}
    1. & A customer can browse available book titles\\
    2. & The layout of the menu can be toggled from a tabbed view to a drop down list.\\
    3. & A particular book name can be searched for\\
    4. & A customer can order a book
  \end{tabularx}
}{
  \begin{tabularx}{0.8\textwidth}{lX}
     & \\
    2. & Correct. This is just about how the menu options are displayed. No actual function being performed. The others are particular actions the system can perform.\\
     & \\
     & 
  \end{tabularx}
}

%xbd
%[3, 1, 2, 4, 7, 5, 6, 8]

\card{
  \textbf{In a university system, which of these would not be a good domain class?}\\\vspace{1em}
  \begin{tabularx}{0.8\textwidth}{lX}
    1. & Exam\\
    2. & Select Course\\
    3. & Course\\
    4. & Student
  \end{tabularx}
}{
  \begin{tabularx}{0.8\textwidth}{lX}
     & \\
    2. & Correct. This sounds like a use case. The others are typical domain areas of a university.\\
     & \\
     & 
  \end{tabularx}
}

%xbh
%[2, 4, 3, 1, 6, 8, 7, 5]

\card{
  \textbf{Which of these would not be a good domain in a Parcel Delivery System?}\\\vspace{1em}
  \begin{tabularx}{0.8\textwidth}{lX}
     & \\
    2. & Customer Account\\
    3. & Parcel\\
    4. & Van
  \end{tabularx}
}{
  \begin{tabularx}{0.8\textwidth}{lX}
     & \\
     & \\
     & \\
    4. & Correct. A van might be used for the delivery, but wouldn't be used as a domain class. The others are well defined domain areas which the system will want to use.
  \end{tabularx}
}

%xaq
%[1, 4, 3, 2, 5, 8, 7, 6]

\card{
  \textbf{Requirements are ....?}\\\vspace{1em}
  \begin{tabularx}{0.8\textwidth}{lX}
    1. & ... recorded in an official document.\\
    2. & ... gathered at the end of the software process.\\
    3. & ... a waste of time.\\
    4. & ... not recorded at all. They are remembered from conversations.
  \end{tabularx}
}{
  \begin{tabularx}{0.8\textwidth}{lX}
    1. & Correct. Requirements are recorded in a semi-structured document. The others not serious answers at all\\
     & \\
     & \\
     & 
  \end{tabularx}
}

%xbg
%[1, 2, 3, 4, 5, 6, 7, 8]

\card{
  \textbf{In a Bank System, which of these would not be a good domain class?}\\\vspace{1em}
  \begin{tabularx}{0.8\textwidth}{lX}
    1. & Tax System\\
     & \\
    3. & Account\\
    4. & Mortgage
  \end{tabularx}
}{
  \begin{tabularx}{0.8\textwidth}{lX}
    1. & Correct. Tax system would be an external system, a separate system. The others are common domains within a typical bank.\\
     & \\
     & \\
    4. & Incorrect. The correct answer is Tax system, as this would be an external system, a separate system. The others are common domains within a typical bank.Incorrect. The correct answer is Tax system, as this would be an external system, a separate system. The others are common domains within a typical bank.
  \end{tabularx}
}

%xab
%[1, 3, 2, 4, 5, 7, 6, 8]

\card{
  \textbf{In A Banking System, which of these would you consider a functional requirement?}\\\vspace{1em}
  \begin{tabularx}{0.8\textwidth}{lX}
    1. & Customer can create a new account\\
    2. & Customer can view account changes within 1 hour of transactions happening\\
    3. & Customer can make balance transfers online\\
    4. & The system should have the latest security protection
  \end{tabularx}
}{
  \begin{tabularx}{0.8\textwidth}{lX}
    1. & Correct. This is a clear functional aspect of a banking system.\\
    2. & Incorrect. This is more about the performance of a function, rather than the actual function. If it said Customer can view account changes only, then that would be a functional requirement. The correct answer is Customer can create a new account as this is a a clear functional aspect of  a banking system.\\
    3. & Incorrect. This is non-functional as it has the constraint of being online, rather than just the function itself. The correct answer is Customer can create a new account as this is a a clear functional aspect of a banking system.\\
    4. & Incorrect. This is more about the system performance and how it performs, rather than an actual function it does. The correct answer is Customer can create a new account as this is a a clear functional aspect of a banking system.
  \end{tabularx}
}

%xaf
%[1, 2, 3, 4, 5, 6, 7, 8]

\card{
  \textbf{In an Aircraft system which of the following is the odd one out?}\\\vspace{1em}
  \begin{tabularx}{0.8\textwidth}{lX}
    1. & The warning lights for the pilot should be red and flash\\
    2. & There must be a warning light when fuel is low\\
    3. & A warning light must indicate if the cabin door is not shut properly\\
    4. & Pilot should be able to disable a warning light
  \end{tabularx}
}{
  \begin{tabularx}{0.8\textwidth}{lX}
    1. & Correct. This is a non-functional requirement, as it says how the function of the warning light should behave. The others are all functional requirements\\
     & \\
     & \\
     & 
  \end{tabularx}
}

%xbf
%[2, 1, 3, 4, 6, 5, 7, 8]

\card{
  \textbf{In a Car hire system which of these would be a good domain class?}\\\vspace{1em}
  \begin{tabularx}{0.8\textwidth}{lX}
    1. & Car hire service\\
    2. & Car\\
    3. & Add Customer\\
    4. & Purchase new car
  \end{tabularx}
}{
  \begin{tabularx}{0.8\textwidth}{lX}
     & \\
    2. & Correct. A car would be an important domain area of the system.  The others all look like use cases, as they are doing something to things.\\
     & \\
     & 
  \end{tabularx}
}

%xau
%[3, 2, 4, 1, 7, 6, 8, 5]

\card{
  \textbf{Which is the odd one out here?}\\\vspace{1em}
  \begin{tabularx}{0.8\textwidth}{lX}
    1. & Staff member deletes a booking.\\
    2. & Staff member adds a booking.\\
    3. & Staff member changes a booking.\\
    4. & Staff member manages hotel bookings.
  \end{tabularx}
}{
  \begin{tabularx}{0.8\textwidth}{lX}
     & \\
     & \\
     & \\
    4. & Correct. This is too general. The others are of more fine granularity and refer to a specific task within the system.
  \end{tabularx}
}

%xba
%[2, 3, 4, 1, 6, 7, 8, 5]

\card{
  \textbf{Which of these is not a very good use case?}\\\vspace{1em}
  \begin{tabularx}{0.8\textwidth}{lX}
    1. & Review customer records.\\
    2. & Read customer records.\\
    3. & Open customer records.\\
    4. & Customer records are reviewed.
  \end{tabularx}
}{
  \begin{tabularx}{0.8\textwidth}{lX}
     & \\
     & \\
     & \\
    4. & Correct. This is not in a very good format, its noun (customer) then verb (reviewed) The others are of a better format as they say what we doing before the actual object.
  \end{tabularx}
}

%xax
%[3, 1, 2, 4, 7, 5, 6, 8]

\card{
  \textbf{In a James Bond style car system, which of these would not be a use case?}\\\vspace{1em}
  \begin{tabularx}{0.8\textwidth}{lX}
    1. & Spray oil out of car\\
    2. & Windows should be black, so no one can see in.\\
    3. & Eject passenger seat.\\
    4. & Boost  car speed
  \end{tabularx}
}{
  \begin{tabularx}{0.8\textwidth}{lX}
    1. & Incorrect.  This is  a clear use case in the correct format.  spray(verb-direct) oil out of car(object-noun). The correct answer is Windows should be black, so no one can see in as this is more of a requirement of the car, not an action performed by the car in the format of verb-direct  - object-noun,  such as, Eject (doing, verb-direct) passenger seat (to something, object-noun).\\
    2. & Correct. This is more of a requirement of the car, not an action performed by the car in the format of verb-direct  - object-noun,  such as, Eject (doing, verb-direct) passenger seat (to something, object-noun).\\
    3. & Incorrect.  This is  a clear use case in the correct format.  Eject (verb-direct) Passenger seat (object-noun). The correct answer is Windows should be black, so no one can see in as this is more of a requirement of the car, not an action performed by the car in the format of verb-direct  - object-noun,  such as, Eject (doing, verb-direct) passenger seat (to something, object-noun).\\
    4. & Incorrect.  This is  a clear use case in the correct format.  Boost(verb-direct) car speed(object-noun). The correct answer is Windows should be black, so no one can see in as this is more of a requirement of the car, not an action performed by the car in the format of verb-direct  - object-noun,  such as, Eject (doing, verb-direct) passenger seat (to something, object-noun).
  \end{tabularx}
}

%xao
%[1, 4, 3, 2, 5, 8, 7, 6]

\card{
  \textbf{Which is the odd one out from these requirements for a music download website system?}\\\vspace{1em}
  \begin{tabularx}{0.8\textwidth}{lX}
    1. & Members can manage their account.\\
    2. & Members can leave comments on artists page.\\
    3. & Artists can add songs to their catalogue.\\
    4. & Members can download bought songs.
  \end{tabularx}
}{
  \begin{tabularx}{0.8\textwidth}{lX}
    1. & Correct. This is not at the same abstraction level as the others.  It is of a more general nature about what the whole system should achieve, rather than the more specific functionality of the others.\\
     & \\
     & \\
     & 
  \end{tabularx}
}

%xas
%[2, 4, 3, 1, 6, 8, 7, 5]

\card{
  \textbf{In the hairdressing analogy, which would be a non-functional requirement?}\\\vspace{1em}
  \begin{tabularx}{0.8\textwidth}{lX}
    1. & hair must be cut short round the back\\
    2. & Don't cut too much off the top.\\
    3. & hair must be cut short round the sides\\
    4. & The chair height should set high
  \end{tabularx}
}{
  \begin{tabularx}{0.8\textwidth}{lX}
     & \\
     & \\
     & \\
    4. & Correct. This does not effect the function of the hair being cut. The others are specific to the function of cutting hair
  \end{tabularx}
}

%xay
%[4, 1, 2, 3, 8, 5, 6, 7]

\card{
  \textbf{ How does Craig Larman describe use cases?}\\\vspace{1em}
  \begin{tabularx}{0.8\textwidth}{lX}
    1. & Things the boss does\\
    2. & Things you say you're doing when the boss shows up\\
    3. & Things you do when the boss is away\\
    4. & Things the boss asks you to do
  \end{tabularx}
}{
  \begin{tabularx}{0.8\textwidth}{lX}
    1. & Incorrect. The boss doesn't do everything !!  Larman describes them as Things you say you're doing when the boss shows up\\
    2. & Correct.  Things you say you're doing when the boss shows up, so hes basically saying use cases are things the boss would expect you to be doing. Example:  Boss: What are you doing?  You: Printing accounts.  Use case format:  Printing (direct-verb) accounts (object-noun)\\
    3. & Incorrect. You may well be doing worthwhile things while the boss is away, but Larman describes them as  Things you say you're doing when the boss shows up\\
    4. & Incorrect. This does not sound like too bad a description of what use cases are, but maybe boss should have to tell you what to do and so Larman describes  them as Things you say you're doing when the boss shows up
  \end{tabularx}
}

%xap
%[4, 2, 3, 1, 8, 6, 7, 5]

\card{
  \textbf{Which of these statements, best describes the requirements process?}\\\vspace{1em}
  \begin{tabularx}{0.8\textwidth}{lX}
    1. & In the generic software development process, Requirements comes after domain modelling to aid the system class design.\\
    2. & Requirements are essential to deciding what software architecture to use.\\
    3. & Requirements tell the software developer everything they need to know to develop the required system.\\
    4. & Requirements are an essential input to the software development process.
  \end{tabularx}
}{
  \begin{tabularx}{0.8\textwidth}{lX}
    1. & Incorrect. Domain modelling can not be done without any requirements gathering and so should be done before this. The answer is Requirements are an essential input to the software development process. Remember the hairdressing analogy where the hairdresser must know how the customer wants their hair cut. The haircut Requirements.\\
    2. & Incorrect. They could play some part in this decision, but by no means essential. The answer is Requirements are an essential input to the software development process. Remember the hairdressing analogy where the hairdresser must know how the customer wants their hair cut. The haircut Requirements.\\
    3. & Incorrect. They play a huge part in what the the developer needs to know. However it is not everything. Activity diagrams, observations, interviews, documentation amongst other things can help aid this also. The answer is Requirements are an essential input to the software development process. Remember the hairdressing analogy where the hairdresser must know how the customer wants their hair cut. The haircut Requirements.\\
    4. & Correct. Requirements come in at the start to help the development process begin. Remember the hairdressing analogy where the hairdresser must know how the customer wants their hair cut. The haircut Requirements.
  \end{tabularx}
}

%xat
%[1, 3, 2, 4, 5, 7, 6, 8]

\card{
  \textbf{Which of the following would be a sensible use case in a banking system?}\\\vspace{1em}
  \begin{tabularx}{0.8\textwidth}{lX}
    1. & Customer withdraws money.\\
    2. & Customers should receive an account statement every month\\
    3. & Customer records should be stored in a  back up database.\\
    4. & Customer manages account.
  \end{tabularx}
}{
  \begin{tabularx}{0.8\textwidth}{lX}
    1. & Correct. This is a clear use case in the format of verb-direct object noun - Withdraws Money, with the Customer being the actor performing this.\\
    2. & Incorrect. This seems like more of a requirements, a use case equivalent would be Send account statement, which follows the use case format of verb-direct object noun.  Send (verb-direct) account statement (noun). The answer is Customer withdraws money.  In this case it is Customer (Actor) withdraws (verb-direct) money (object-noun)\\
    3. & Incorrect. This seems like more of a requirements, a use case equivalent would be Back up customer records, which follows the use case format of verb-direct object noun.  Back up (verb-direct) Customer records (noun). The answer is Customer withdraws money.  In this case it is Customer (Actor) withdraws (verb-direct) money (object-noun)\\
    4. & Incorrect. This is a very general use case and would comprised of many smaller, more fine grain use cases, such as Transfer Money, View account and withdraw money. The answer is Customer withdraws money.  In this case it is Customer (Actor) withdraws (verb-direct) money (object-noun)
  \end{tabularx}
}

%xbk
%[4, 3, 1, 2, 8, 7, 5, 6]

\card{
  \textbf{Which of these is not a good domain class for a game system?}\\\vspace{1em}
  \begin{tabularx}{0.8\textwidth}{lX}
     & \\
     & \\
    3. & Box game comes in.\\
    4. & Players Score
  \end{tabularx}
}{
  \begin{tabularx}{0.8\textwidth}{lX}
     & \\
     & \\
    3. & Correct. This is not important in the game system. The others sound like things used in a game.\\
     & 
  \end{tabularx}
}

%xbi
%[1, 2, 4, 3, 5, 6, 8, 7]

\card{
  \textbf{How can a domain class play a part in many use cases?}\\\vspace{1em}
  \begin{tabularx}{0.8\textwidth}{lX}
    1. & It may have different roles in them or the same role.\\
    2. & It will always have the same role in them.\\
    3. & Each domain class represents the whole system, so by definition they do.\\
    4. & They always play a part in every use case.  They just do.
  \end{tabularx}
}{
  \begin{tabularx}{0.8\textwidth}{lX}
    1. & Correct. They can play different roles in different use cases.\\
    2. & Incorrect. They do not always have the same role in all use cases, they may have a different one or not play a role in some use cases at all.\\
    3. & Incorrect. Each domain class does not represent the whole system, they represent a domain within the system and so can play a role in many use cases, but they could only play a role in one.\\
    4. & Incorrect. They do not necessarily play the a part in every use case, they may only play a role in one.
  \end{tabularx}
}

%xah
%[4, 2, 3, 1, 8, 6, 7, 5]

\card{
  \textbf{Which of the following would be a functional requirement in a James Bond style car?}\\\vspace{1em}
  \begin{tabularx}{0.8\textwidth}{lX}
    1. & The gadgets should have new software updates done every month\\
    2. & Would be good to have the controls light up in fancy colours\\
    3. & The stereo should be able to play extra loud\\
    4. & Must be able to spray oil out the back to stop any baddies chasing him
  \end{tabularx}
}{
  \begin{tabularx}{0.8\textwidth}{lX}
    1. & Incorrect. Updating the system is more of a preference and not a functional requirement The correct answer is Must be able to spray oil out the back to stop any baddies chasing him Because this states a function it should perform.\\
    2. & Incorrect. This is a preference about how it should look, not an actual action it performs. The odd one out is Must be able to spray oil out the back to stop any baddies chasing him Because this states a function it should perform.\\
    3. & Incorrect. This is a preference about how the stereo should perform its function, not the actual function itself. The correct answer is Must be able to spray oil out the back to stop any baddies chasing him Because this states a function it should perform.\\
    4. & Correct. This states a function it should perform.
  \end{tabularx}
}

%xbb
%[1, 2, 4, 3, 5, 6, 8, 7]

\card{
  \textbf{A Domain Class can be used to ... ?}\\\vspace{1em}
  \begin{tabularx}{0.8\textwidth}{lX}
    1. & Realise several use cases.\\
    2. & Realise only one use case\\
    3. & discover use cases\\
    4. & aid requirements gathering
  \end{tabularx}
}{
  \begin{tabularx}{0.8\textwidth}{lX}
    1. & Correct.  One Domain class can play a role in many different use cases. For example in an ATM example. Imagine a domain class called display, you would expect this to play a role in achieving the following use cases: Display Balance and Withdraw Cash (to display amounts to withdraw)\\
    2. & Incorrect.  A Domain class can help realise several use cases, not just one.For example in an ATM example. Imagine a domain class called display, you would expect this to play a role in achieving the following use cases: Display Balance and Withdraw Cash (to display amounts to withdraw)\\
    3. & Incorrect. Use cases will already have been discovered before, from requirements gathering, amongst other techniques\\
    4. & Incorrect. Requirements gathering would be done a while before domain classes are used.  Remember the order of the phases:  - Requirements - Use cases - domain modelling (which involved domain classes) The answer is A Domain class can be used to realise several use cases.
  \end{tabularx}
}

%xac
%[1, 3, 2, 4, 5, 7, 6, 8]

\card{
  \textbf{Which of these requirements for a hotel booking system is the odd one out?}\\\vspace{1em}
  \begin{tabularx}{0.8\textwidth}{lX}
    1. & A customer should be able to manage their bookings\\
    2. & A customer can cancel a room booking.\\
    3. & A customer can book a room up to 12 months in advance.\\
    4. & A customer can amend a booking.
  \end{tabularx}
}{
  \begin{tabularx}{0.8\textwidth}{lX}
    1. & Correct. The level of abstraction is higher than the others. It is very general and can include many other things.  The others are of a more fine granularity and refer to more specific requirements\\
     & \\
     & \\
     & 
  \end{tabularx}
}

%xan
%[1, 3, 2, 4, 5, 7, 6, 8]

\card{
  \textbf{In an ATM System, which would be a non-functional requirement?}\\\vspace{1em}
  \begin{tabularx}{0.8\textwidth}{lX}
    1. & The interface should be physically robust and hard wearing\\
    2. & Customer can select an amount to withdraw.\\
    3. & Customer can view their account balance.\\
    4. & Cash must be ejected for the customer.
  \end{tabularx}
}{
  \begin{tabularx}{0.8\textwidth}{lX}
    1. & Correct. This is a constraint on the performance of the interface, not an actual function the system performs.The others are simple, yet essential functional activities you'd expect from an ATM and so, are functional activities.\\
     & \\
     & \\
     & 
  \end{tabularx}
}

%xbj
%[3, 1, 4, 2, 7, 5, 8, 6]

\card{
  \textbf{Why is DATABASE not a good domain class name?}\\\vspace{1em}
  \begin{tabularx}{0.8\textwidth}{lX}
    1. & Its in capitals.\\
    2. & It does not say what the database is holding, CUSTOMER DATABASE would be better.\\
    3. & Its external to the system.\\
    4. & Databases can not be used as domain classes.
  \end{tabularx}
}{
  \begin{tabularx}{0.8\textwidth}{lX}
    1. & Incorrect. It has nothing to do with it being in capitals. The problem with this one is it needs more description, such as CUSTOMER DATABASE, to give it more meaning.\\
    2. & Correct. Its more description, on its own, it means nothing.\\
    3. & Incorrect. It could be a big part of the system, but the problem with this one is it needs more description, such as CUSTOMER DATABASE, to give it more meaning.\\
    4. & Incorrect. Databases can be very good domain classes, but the problem with this one is it needs more description, such as CUSTOMER DATABASE, to give it more meaning.
  \end{tabularx}
}

%xai
%[4, 1, 3, 2, 8, 5, 7, 6]

\card{
  \textbf{In a flight booking database system, what would be a non-functional requirement?}\\\vspace{1em}
  \begin{tabularx}{0.8\textwidth}{lX}
    1. & A particular booking information can be read from the database\\
    2. & The booking database should be backed up every 5 minutes\\
    3. & A cancelled booking can be deleted from the database\\
    4. & A new booking should be added to the database
  \end{tabularx}
}{
  \begin{tabularx}{0.8\textwidth}{lX}
     & \\
    2. & Correct. this is more of a safety requirement and not a actual functional requirement. The others are actual functions on the database\\
     & \\
     & 
  \end{tabularx}
}

%xaa
%[2, 4, 1, 3, 6, 8, 5, 7]

\card{
  \textbf{In a Banking System, which of these would you consider a non-functional requirement?}\\\vspace{1em}
  \begin{tabularx}{0.8\textwidth}{lX}
    1. & Customer can withdraw Cash\\
    2. & Customer can close their account\\
    3. & Cash withdraws should be possible 24/7\\
    4. & Customer can transfer money to another account
  \end{tabularx}
}{
  \begin{tabularx}{0.8\textwidth}{lX}
     & \\
     & \\
    3. & Correct. This is how something should be done, not what should be done.\\
     & 
  \end{tabularx}
}

%xar
%[3, 2, 1, 4, 7, 6, 5, 8]

\card{
  \textbf{In a stock ordering system which of these is the odd one out?}\\\vspace{1em}
  \begin{tabularx}{0.8\textwidth}{lX}
    1. & The system should be able to display stock levels.\\
    2. & The system must be able to order a given stock item.\\
    3. & The system should help the company manage its stock levels.\\
    4. & Stock items should be displayed with a picture.
  \end{tabularx}
}{
  \begin{tabularx}{0.8\textwidth}{lX}
     & \\
     & \\
    3. & Correct. This is a very general requirement, where as the rest are more fine grain requirements.\\
     & 
  \end{tabularx}
}

%xae
%[4, 2, 3, 1, 8, 6, 7, 5]

\card{
  \textbf{At what stage of the generic software developed process does Requirements come?}\\\vspace{1em}
  \begin{tabularx}{0.8\textwidth}{lX}
    1. & in the middle.\\
    2. & Second, after Domain Modelling and before System Design\\
    3. & Last\\
    4. & At the start
  \end{tabularx}
}{
  \begin{tabularx}{0.8\textwidth}{lX}
     & \\
     & \\
     & \\
    4. & Correct. Requirements are done at the start, generally speaking, however it is best practice for them to be done all the time, to adapt to any changes and correct any you may have wrong. But mostly done at the start.
  \end{tabularx}
}

%xav
%[4, 3, 1, 2, 8, 7, 5, 6]

\card{
  \textbf{In a spaceship system, which use case would be the odd one out?}\\\vspace{1em}
  \begin{tabularx}{0.8\textwidth}{lX}
    1. & Spaceman creates an error log\\
    2. & Ground control reads spaceship location\\
    3. & Spaceship takes off.\\
    4. & Spaceman increases speed.
  \end{tabularx}
}{
  \begin{tabularx}{0.8\textwidth}{lX}
     & \\
     & \\
    3. & Correct. this is a very general function, which is likely to involve many other use cases, such as open rocket thrusters for example.  The others are specific use cases.\\
     & 
  \end{tabularx}
}

%xbc
%[4, 2, 1, 3, 8, 6, 5, 7]

\card{
  \textbf{Which of these would not be a domain class in an ATM Cash machine system?}\\\vspace{1em}
  \begin{tabularx}{0.8\textwidth}{lX}
    1. & screen\\
    2. & dispenser\\
    3. & ATM\\
    4. & keypad
  \end{tabularx}
}{
  \begin{tabularx}{0.8\textwidth}{lX}
     & \\
     & \\
    3. & Correct.  This is the name of the system itself not domains within it. The others are domains which play their own individual parts in the ATM System.\\
     & 
  \end{tabularx}
}

%xbe
%[3, 2, 1, 4, 7, 6, 5, 8]

\card{
  \textbf{In the generic software engineering process, domain modelling comes ....?}\\\vspace{1em}
  \begin{tabularx}{0.8\textwidth}{lX}
    1. & after System design.\\
    2. & before requirements.\\
    3. & After requirements.\\
    4. & After testing.
  \end{tabularx}
}{
  \begin{tabularx}{0.8\textwidth}{lX}
     & \\
     & \\
    3. & Correct. The order is: Requirements, Domain Modelling, System Modelling, Implementation...\\
     & 
  \end{tabularx}
}

%xaz
%[4, 1, 2, 3, 8, 5, 6, 7]

\card{
  \textbf{Which of these is a good use case?}\\\vspace{1em}
  \begin{tabularx}{0.8\textwidth}{lX}
    1. & Database update\\
    2. & Add Client Record\\
    3. & Client Database\\
    4. & Client update
  \end{tabularx}
}{
  \begin{tabularx}{0.8\textwidth}{lX}
    1. & Incorrect. It is not clear what this is, are we updating a database? And what database?  Is it a database updating something else? If so what? The answer is Add client record, which is a typical use case with the format of verb-direct  - object-noun .  Add (verb-direct) Client Record (object-noun).\\
    2. & Correct. This is a typical use case with the format of verb-direct  - object-noun .  Add (verb-direct) Client Record (object-noun).\\
    3. & Incorrect. This sounds like a domain object rather than a use case, its not doing anything, its just the name of something in the system. The answer is Add client record, which is a typical use case with the format of verb-direct  - object-noun .  Add (verb-direct) Client Record (object-noun).\\
    4. & Incorrect. It is not clear what this is, are we updating a client? Is it a client updating something? If so what? The answer is Add client record, which is a typical use case with the format of verb-direct  - object-noun .  Add (verb-direct) Client Record (object-noun). 
  \end{tabularx}
}

%xag
%[3, 4, 2, 1, 7, 8, 6, 5]

\card{
  \textbf{In a spaceship control system, which of the following is the odd one out?}\\\vspace{1em}
  \begin{tabularx}{0.8\textwidth}{lX}
    1. & The system must have an autopilot mode to guide it to a given position.\\
    2. & The system should adjust pressure within the spaceship.\\
    3. & The system should inform ground control its exact position, when requested\\
    4. & The system should provide a complete flight management solution.
  \end{tabularx}
}{
  \begin{tabularx}{0.8\textwidth}{lX}
     & \\
     & \\
     & \\
    4. & Correct. This is not at the same granularity as the others. it is very general and does not state any specific requirement. Where as the others are more fine grain, referring to exact  requirements.
  \end{tabularx}
}

%xak
%[4, 2, 1, 3, 8, 6, 5, 7]

\card{
  \textbf{Imagine an e-commerce system, which could be a functional requirement?}\\\vspace{1em}
  \begin{tabularx}{0.8\textwidth}{lX}
    1. & Pictures of products on the website should be displayed in colour\\
    2. & Customer Data should comply with the Data Protection Act.\\
    3. & Customer can add an item to their basket.\\
    4. & Address labels should be printed on sticky paper.
  \end{tabularx}
}{
  \begin{tabularx}{0.8\textwidth}{lX}
    1. & Incorrect. it is a preference on how pictures of products should be displayed. Remember the car analogy, the car colour made no effect what so ever of the its functional requirements.  The correct answer is Customer can add an item to their basket as this is a sensible basic functionality you could expect of an e-commerce system.\\
    2. & Incorrect. This is a condition on how customer data should be treated. not a function itself. The correct Answer is Customer can add an item to their basket as this is a sensible basic functionality you could expect of an e-commerce system.\\
    3. & Correct. This is a sensible basic functionality you could expect of an e-commerce system. The others are all some kind of constraints on how the business performs tasks and acts.\\
    4. & Incorrect. If it was just Must be able to print Address labels It could be seen as a functional requirement However, it is a condition on this, that the labels should be on sticky paper.  The correct answer is Customer can add an item to their basket as this is a sensible basic functionality you could expect of an e-commerce system.
  \end{tabularx}
}

%xad
%[1, 2, 3, 4, 5, 6, 7, 8]

\card{
  \textbf{Which one of these requirements for a car hire company is the odd one out? }\\\vspace{1em}
  \begin{tabularx}{0.8\textwidth}{lX}
    1. & The system should allow clients with a way to hire cars\\
    2. & Clients should be able to browse the vehicles available to hire\\
    3. & Clients should be able to view vehicle availability\\
    4. & Clients should be able to extend the time period of a vehicle hire, availability permitting.
  \end{tabularx}
}{
  \begin{tabularx}{0.8\textwidth}{lX}
    1. & Correct. This one is too general and not of the same granularity as the others.\\
     & \\
     & \\
     & 
  \end{tabularx}
}

%xal
%[2, 4, 3, 1, 6, 8, 7, 5]

\card{
  \textbf{Imagine an e-commerce system, which would be a non-functional requirement?}\\\vspace{1em}
  \begin{tabularx}{0.8\textwidth}{lX}
    1. & Customer can change their payment details\\
    2. & Orders can be cancelled\\
    3. & Customer can view their order status.\\
    4. & Orders should be confirmed via email within 24 hours.
  \end{tabularx}
}{
  \begin{tabularx}{0.8\textwidth}{lX}
     & \\
     & \\
     & \\
    4. & Correct. The time constraint on this makes it non-functional. The others are clear functions, with no constraints on how they are done.  Remember the file download example, the constraints on speed of download, made no effect on the file being downloaded or uploaded.
  \end{tabularx}
}

%xam
%[3, 1, 4, 2, 7, 5, 8, 6]

\card{
  \textbf{Imagine a Hotel Management System, which would be a good example of a functional requirement?}\\\vspace{1em}
  \begin{tabularx}{0.8\textwidth}{lX}
    1. & The system should have the latest anti-virus software installed.\\
    2. & A room can be reserved for a customer.\\
    3. & Each room should be described clearly on the hotel website.\\
    4. & The database should have enough storage for 10,000 customer records.
  \end{tabularx}
}{
  \begin{tabularx}{0.8\textwidth}{lX}
     & \\
    2. & Correct. This is basic functional requirement with nothing about how it should do it. The others have some description properties of the system or how functions should be performed.\\
     & \\
     & 
  \end{tabularx}
}

