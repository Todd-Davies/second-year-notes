\card{
  What are the two types of virtualisation?
}{
  System virtualisation (run whole OS inside software e.g. VMware) and process
  virtualisation (run a process under a control layer of software, e.g. JVM).
}

\card{
  What are the three main advantages of virtualisation?
}{
  Translation (between instruction sets, system API's etc), abstraction
  (providing garbage collection, debugging etc), or multiplexing (e.g. RAID or
  emulating CD drives).
}

\flashcard{
  A hypervisor runs in \blank{privileged} mode, and can run virtual machines in
  a \blank{unprivileged} mode, having \blank{traps} for when the guest OS does
  something that requires \blank{system privileges}.
}

\card{
  What happens when you start a VM?
}{
  \begin{tabular}{lr}
    $\cdot$ & Save the current registers\\
    $\cdot$ & Load the VM registers\\
    $\cdot$ & Move the PC to the start address of the VM
  \end{tabular}
}

\card{
  When is it best to stop a VM?
}{
  When the VM's IO is quiescent (i.e. not doing anything).
}

\card{
  What is retained when a VM is stopped/paused?
}{
  Memory, IO state, CPU registers, open files, network connections etc
}

\card{
  What operations can we do on a VM?
}{
  \begin{tabular}{lr}
    $\cdot$ & Move VM's between machines (live migration)\\
    $\cdot$ & Take a snapshot of a VM\\
    $\cdot$ & Restore a VM from a snapshot (quickly)\\
    $\cdot$ & Load balancing using live migration
  \end{tabular}
}

\card{
  What are the two phases in live migration?
}{
  The warm up phase and the stop and copy phase.
}