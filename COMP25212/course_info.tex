\section*{Introduction}

The basic architecture of computer systems has been covered in first year course
units which detailed both the instruction set architecture and the micro-
architecture (hardware structure) of simple processors. Although these principle
underlie the vast majority of modern computers, there are a wide range of both
hardware and software techniques which are employed to increase the performance,
reliability and flexibility of systems.

\section*{Aims}

The aims of this course are to introduce the most important system architecture
approaches. To give a wider understanding of how real systems operate and, from
that understanding, the ability to optimise their use.

The syllabus includes:

\begin{itemize}
	\item The motivation behind advanced architectural techniques.
	\item Caching
	\item The need to overcome latency. Caching as a principle, examples of caching in practice. Processor cache structure and operation.
	\item Pipelining
	\item Principles of pipelining. Implementation of a processor pipeline and its properties. Pipelining requirements and limitations. Additional support for pipelining.
	\item Multi-Threading
	\item Basic multi-threading principles. Processor support for multi-threading. Simultaneous multi-threading.
	\item Multi-Core
	\item Motivation for multi-core. Possible multi-core structures. Cache coherence.
	\item File System Support
	\item Implementation of file systems. RAID
	\item Virtual Machines
	\item Motivation for Virtual Machines. Language Virtual Machines. System Virtual Machines. Virtual Machine implementation. Binary Translation
\end{itemize}