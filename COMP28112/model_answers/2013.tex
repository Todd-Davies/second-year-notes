\answer{Explain briefly what is meant by the term middleware.}{2}{2013.1.a}

Software that sits in between client applications and the operating system that
abstracts the details of implementing programming tasks and masks the
heterogeneity of the underlying platforms from the client application.

\answer{Explain briefly what failures are known as Byzantine
failures.}{2}{2013.1.b}

When a remote system either:

\begin{itemize}
  \item Does not reply to messages
  \item Sends faulty replies to message with miscellaneous data
  \item Sends maliciously crafted messages
  \item Acts inconsistently when interacting with different components
\end{itemize}

\answer{Describe briefly the two-phase commit protocol.}{2}{2013.1.c}

When a client wants to commit, it sends a message to a server. The server then
asks every client whether they are ready for a commit. Each client replies with
a yes/no. If there is a no, then a \texttt{global\_abort} message is sent,
otherwise a \texttt{global\_commit} message is sent. Upon receiving a global
commit or abort message, each client does exactly that, which ensures that only
one operation is done for all clients.

\answer{What is meant when a service is provided with at least once
semantics?}{2}{2013.1.d}

The application may send the RPC call multiple times before it receives word
from the server that the RPC call has taken place and was successful. This may
result in the RPC call being executed multiple times.

\answer{Why is it practically impossible to achieve exact synchronisation of
clocks in a distributed system?}{2}{2013.1.d}

The latency in a network is variable, so you can never know exactly how long a
message took to reach a server. Because of this, if one server sends its current
clock to another server, the latter one won't know how long it took for the
message to reach it and therefore it won't know how much to increment the clock
by.

Methods such as Cristian's algorithm and the Berkeley algorithm attempt to solve
this problem, but can only do so with some assumptions and to a limited
accuracy.

\answer{When using Java RMI, what is the purpose of the RMI
registry?}{2}{2013.1.f}

To act as a centeral repository for computers to access remote objects. Clients
can interrogate the registry using a string and get an object back. This allows
RMI methods to have parameters that are strings that point to objects in the
repository so you can pass arbitrary data structures and objects easily.

\answer{What is meant by parameter marshalling?}{2}{2013.1.g}

When an RPC call is made, the parameters cannot be sent directly (since they may
be pointers etc), so they must be serialized into a blob of text or binary data
before they are sent (marshalled) and deserialised at the other end by the
server (unmarshalled).

\answer{What is the key difference between caching and
replication?}{2}{2013.1.h}

Replication is keeping two fully fledged entities up to date with each other in
order to provide redundancy or distribute the load of a service over more
machines.

Caching is merely a small piece of hardware or a small program that remembers
the results of expensive computations after they've been done once and returns
the result without doing any computation if the same request comes again. Rarely
is there any business logic in a cache.

\answer{Explain briefly what Little’s Law is.}{2}{2013.1.i}

Little's law dictates that the average number of elements in a queue is:

\[
  \text{av. number} = \text{av. time between arrivals} \times \text{time to process one}
\]

This allows us to estimate how a system will handle load and what its queue size
will be (and if its appropriate).

\answer{In the context of lab exercise 2, what would you do to launch a denial
of service attack against the server?}{2}{2013.1.j}