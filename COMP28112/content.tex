% Set the author and title of the compiled pdf
\hypersetup{
	pdftitle = {\Title},
	pdfauthor = {\Author}
}

\section{Distributed computing}

A distributed system is a computing platform build with many computers that:

\begin{itemize}
  \item Operate concurrently,
  \item Are physically distributed (and can fail independantly)
  \item Are linked by a network
  \item Have independent clocks
\end{itemize}

Leslie Lamport once said that:

%TODO: Quote thing

You know you have a distributed system when the crash of a computer you've never
heard of stops you from getting work done.

Distributed systems have evolved from sinple systems in the 70's and 80's. Early
systems were for banks and airline booking systems, but the real proliferation
of the technique arose with the internet in the early 90's.

%TODO: The above is supposed to be lecture 1. Improve on it!

% Lecture two

% TODO: How to calculate latency
 \section{Parellalising processes}

Many applications can be parellised by doing homogenous operations on different
processors on different data. If this is the case, in ideal conditions, your
speedup will be the same as the number of processors you're using as opposed to
using just one processor.

Unfortunately for us, the speedup is not linear, since it takes time to split
the data, coordinate the machines and collate the results. There is also a limit
to how many processors will keep the speed improving or even keeping constant.
If we have more processors than we can actually use, then the overhead of
managing them will probably decrease performance, since the'll be doing nothing
useful.
